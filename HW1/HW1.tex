\documentclass [10pt]{article}
\textheight	8.7in
\textwidth	6.5in
\topmargin	    0in
\oddsidemargin  0in
\evensidemargin 0in
\baselineskip 15pt
\usepackage{amssymb,amsmath,amstext}
\usepackage{amsfonts}
\usepackage{algpseudocode}
\begin{document}
\title{Theory of Computation Assignment no. 1}
\author{Goktug Saatcioglu}
\date{}
\maketitle
\begin{enumerate}
	\item[(1)]If $x = True$ and $y = True$, then the LHS equals $False$ since $\neg(True \land True) = \neg True = False$. The RHS equals $(\neg True \lor \neg True) = False \lor False = False$, which is equal to the LHS. Thus, when both $x$ and $y$ are $True$, the LHS $=$ RHS.\\ When $x = True$ or $y = True$, but not $x = True$ and $y = True$, we see that the LHS equals $True$ since $\neg(True \land False) = \neg False = True$ and $\neg(False \land True) = \neg False = True$. The RHS in both cases equals $True$ because $(\neg True \lor \neg False) = False \lor True = True$ and $(\neg False \lor \neg True) = True \lor False = True$. Thus, when either $x$ or $y$ is $True$ (but not both), the LHS $=$ RHS.\\ Finally, if $x = False$ and $y = False$, then the LHS equals $True$ since $\neg(False \land False) = \neg False = True$. The RHS equals $(\neg False \lor \neg False) = True \lor True = True$, which is equal to the LHS. Thus, when both $x$ and $y$ are $False$, the LHS $=$ RHS.\\ We see that the for every setting for the binary variables $x$ and $y$, the LHS and RHS of the equation evaluate the same truth value. $\therefore \neg(x \land y) = (\neg x \lor \neg y)$.
	\item[(2)]Suppose there are $2n+1$ beads on a necklace. Assign a color to every $2$ beads, this gives us $2n$ beads colored with $n$ colors. By the pigeonhole principle, there is a remaining bead that forms a match with one of the $n$ colors we have. Thus, there is at least a set of at least $3$ beads on the necklace that will have the same color.
	\item[(3)]Let $T(n)$ be the function over natural numbers $n$, defined as follows: $T(1) = 2$ and $T(2) = 4$ and for any other $n$: $$T(n) = 2 + \min_{i=1 \dots n-2} \{T(i) + T(n - i - 1)\}.$$\\ Using strong induction we prove that $T(n) = 2n$ for all $n$.\\
	\textbf{Base case.} $n = 3$.\\ $T(3) = 2 + \min_{i=1 \dots 1} \{T(i) + T(2 - i)\} = 2 + T(1) + T(1) = 2 + 2 + 2 = 6 = 2 \times 3 = 2n \therefore$ holds true for base case.\\
	\textbf{Inductive step.} We show for $k \ge 3$, that if $T(h) = 2h$ for all $h \le k$ then $T(k+1)$ is also true.
		\begin{align}
			T(k+1) &= 2 + \min_{i=1 \dots k+1-2} \{T(i) + T(k + 1 - i - 1)\} \nonumber \\
			&= 2 + \min_{i=1 \dots k-1} \{T(i) + T(k - i)\} \label{eq1}
		\end{align}
	In (\ref{eq1}) we see that the $min$ operator attempts to find the minimum between the range $i=1$ to $k-1$ of $T(i) + T(k - i)$. For all $i$ in that range, each iteration of the $min$ operator will evaluate the value of $T(k)$ since by the induction hypothesis, $T(i) + T(k-i) = 2i + 2(k-i) = 2i + 2k - 2i = 2k = T(k)$ for all $i, k$. Thus, the $min$ operator in (\ref{eq1}) will always evaluate to the value of $T(k)$. From the induction hypothesis, we see that $T(k) = 2k$ and thus equation (\ref{eq1}) evaluates to:
		\begin{align} 
			T(k+1) &= 2 + \min_{i=1 \dots k-1} \{T(i) + T(k - i)\} \nonumber \\
			&= 2 + T(k) && \text{[by evaluating the min operator]} \nonumber \\
			&= 2 + 2k && \text{[by the induction hypothesis]} \nonumber \\
			&= 2 (k + 1). \label{eq2}
		\end{align}
	\textbf{Conclusion.} By the principle of strong induction, we see that for $n \ge 3$, $T(n) = 2n$. Furthermore, by definition, $T(1) = 2$ and $T(2) = 4$. $\therefore T(n) = 2n$ for all $n$.
	\item[(4)]Consider the algorithm $E$, given below, which uses $H$ as a subroutine.
		\begin{algorithmic}
			\Function{E}{x}
				\State $A \gets H(x,xx)$
				\If{$A="Yes"$} loop forever \Else \: ($A = "No"$ and) $D(x)$ outputs "Yes" \EndIf
			\EndFunction
		\end{algorithmic}
	If $x$ is not a valid program or $x$ is not a double then $H(x,x)$ evaluates to "Yes". However, to consider valid programs with inputs that are doubles, we must modify $H$ used to determine $A$ such that $H$ gets a valid program $x$, which may or not be a double, and a string representation of $x$ which is a double, making the second input $xx$. Here if the program string were to be a double, such as $x=yy$, then $xx=yyyy$ is still a double and the program logic is held ($H$ still attempts to solve the 2HALT problem).\\
	Now consider what happens if we run $E$ on its own code ($E(E)$). We then get the following contradiction:
		\begin{align}
		\text{$E(E)$ never halts} &\implies H(E,EE) = "Yes" \nonumber \\
		&\implies \text{$E(E)$ halts} \implies H(E,EE) = "No" \nonumber \\
		&\implies \text{$E(E)$ never halts} \implies \dots \nonumber
		\end{align}
	From the contradiction we see that there is no algorithm $H(P,x)$ that solves 2HALT. (Note: if we change $H(x,xx)$ to $H(x,x)$ we will obtain the same contradiction. However, now the algorithm $E$ will not be attempting to solve the 2HALT problem.)
\end{enumerate}
\end{document}